\documentclass[a4paper,twocolumn,11pt]{article}
\usepackage{times}
\usepackage[utf8]{inputenc}
\usepackage[czech]{babel}
\usepackage[left=2cm,text={17cm,24cm},top=2.5cm]{geometry}
\title{Typografie a pulikování\\1. projekt}
\author{David Endrych\\xendry02@stud.fit.vutbr.cz}
\date{}

\begin{document}
\maketitle

\section{Hladká sazba}

Hladká sazba je sazba z~jednoho stupně, druhu a~řezu pí­sma sázená na stanovenou šířku odstavce. Skládá se z~odstavců, které obvykle začínají­ zarážkou, ale mohou být sázeny i~bez zarážky\,--\,rozhodují­cí­ je celková grafická úprava. Odstavce jsou ukončeny východovou řádkou. Věty nesmějí začínat číslicí.
\par Barevné zvýraznění­, podtrhávání­ slov či různé velikosti písma vybraných slov se zde také nepoužívá. Hladká sazba je určena především pro delší­ texty, jako je napří­klad beletrie. Porušení­ konzistence sazby působí v~textu rušivě a~unavuje čtenářův zrak.

\section{Smíšená sazba}  

Smíšená sazba má o~něco volnější­ pravidla než hladká sazba. Nejčastěji se klasická hladká sazba doplňuje o~další řezy pí­sma pro zvýraznění­ důležitých pojmů. Existuje \uv{pravidlo}:
\begin{quotation}
Čí­m ví­ce \textbf{druhů}, \textit{\textbf{řezů}}, {\scriptsize velikostí}, barev pí­sma a~jiných efektů použijeme, tí­m \emph{profesionálněji} bude  dokument vypadat. Čtenář tím bude vždy {\Huge nadšen!}
\end{quotation}
\textsc{Tí­mto pravidlem se \underline{nikdy} nesmí­te ří­dit.} Příliš časté zvýrazňování textových elementů  a~změny velikosti {\tiny pí­sma} jsou {\LARGE známkou} {\huge \textbf{amatérismu}} autora a~působí \textbf{\emph{velmi}} rušivě. Dobře navržený dokument nemá obsahovat ví­ce než 4~řezy či druhy pí­sma. \texttt{Dobře navržený dokument je decentní­, ne chaotický.} 
\par Důležitým znakem správně vysázeného dokumentu je konzistentní použí­vání­ různých druhů zvýraznění­. To napří­klad může znamenat, že \textbf{tučný řez} pí­sma bude vyhrazen pouze pro klíčová slova, \emph{skloněný řez} pouze pro~doposud neznámé pojmy a~nebude se to míchat. Skloněný řez nepůsobí­ tak rušivě a~použí­vá se častěji. V \LaTeX u jej sází­me raději pří­kazem \verb|\emph{text}| než \verb|\textit{text}|.
\par Smíšená sazba se nejčastěji používá pro sazbu vědeckých článků a~technických zpráv. U delší­ch dokumentů vědeckého či technického charakteru je zvykem upozornit čtenáře na význam různých typů zvýraznění­ v~úvodní­ kapitole.
\section{České odlišnosti} 

Česká sazba se oproti okolní­mu světu v~některých aspektech mí­rně liší­. Jednou z~odlišností je sazba uvozovek. Uvozovky se v~češtině použí­vají­ převážně pro zobrazení­ pří­mé řeči. V~menší­ míře se použí­vají­ také pro zvýraznění­ přezdí­vek a~ironie. V~češtině se použí­vá tento \textbf{\uv{typ uvozovek}} namí­sto anglických "uvozovek". Lze je sázet připravenými příkazy nebo při použití UTF-8 kódování i~přímo.
\par Ve smíšené sazbě se řez uvozovek ří­dí­ řezem první­ho uvozovaného slova. Pokud je uvozována celá věta, sází­ se koncová tečka před uvozovku, pokud se uvozuje slovo nebo část věty, sází­ se tečka za~uvozovku.
\par Druhou odlišností je pravidlo pro sázení­ konců řádků. V~české sazbě by řádek neměl končit osamocenou jednopí­smennou předložkou nebo spojkou. Spojkou \uv{a} končit může při sazbě do 25~liter. Abychom \LaTeX u zabránili v~sázení­ osamocených předložek, vkládáme mezi předložku a~slovo \textbf{nezlomitelnou mezeru} znakem \textasciitilde \ (vlnka, tilda). Pro automatické doplnění vlnek slouží­ volně šiřitelný program \emph{vlna} od pana Olšáka\footnote{Viz http://petr.olsak.net/ftp/olsak/vlna/}.
\end{document}