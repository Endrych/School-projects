\documentclass[a4paper,twocolumn,11pt]{article}
\usepackage{times}
\usepackage[utf8]{inputenc}
\usepackage[czech]{babel}
\usepackage[left=1.5cm,text={18cm,25cm},top=2.5cm]{geometry}
\usepackage{amsmath}
\usepackage{amsfonts}
\usepackage{amsthm}

\theoremstyle{definition}
\newtheorem{define}{Definice}[section]
\newtheorem{algor}[define]{Algoritmus}
\theoremstyle{plain}
\newtheorem{sente}{Věta}


\begin{document}
\begin{titlepage}
\begin{center}
\Huge
\textsc{Fakulta informačních technologií \\ Vysoké učení technické v Brně} \\
\vspace{\stretch{0.382}}
Typografie a publikování\,--\,2. projekt \\
Sazba dokumentů a matematických výrazů 
\vspace{\stretch{0.618}}
\end{center}
{\LARGE 2017 \hfill David Endrych}
\end{titlepage}

\section*{Úvod} 
V této úloze si vyzkoušíme sazbu titulní strany, matematických vzorců, prostředí a dalších textových struktur obvyklých pro technicky zaměřené texty, například rovnice (\ref{rov1}) nebo definice \ref{def1} na straně \pageref{def1}.
\par Na titulní straně je využito sázení nadpisu podle optického středu s využitím zlatého řezu. Tento postup byl probírán na přednášce.


\section{Matematický text} 

Nejprve se podíváme na sázení matematických symbolů a~výrazů v~plynulém textu. Pro množinu $V$ označuje $\makebox{card}(V)$ kardinalitu $V$. Pro množinu $V$ reprezentuje $V^*$ volný monoid generovaný množinou $V$ s~operací konkatenace.
Prvek identity ve volném monoidu $V^*$ značíme symbolem $\varepsilon$
Nechť $V^+ =V^* - \{\varepsilon\}$ Algebraicky je tedy $V^+$ volná pologrupa generovaná množinou $V$ s~operací konkatenace.
Konečnou neprázdnou množinu $V$ nazvěme \emph{abeceda}.
Pro $\omega \in V^*$ označuje $|\omega|$ délku řetězce $\omega$ Pro $W \subseteq V$ označuje $\mbox{occur}(\omega,W)$ počet výskytů symbolů z~$W$ v~řetězci $\omega$ a~sym$(\omega,i)$ určuje $i$-tý symbol řetězce $\omega$; například sym$(abcd,3) = c$.
\par Nyní zkusíme sazbu definic a~vět s~využitím balíku \texttt{amsthm}.
\begin{define} \label{def1}
\emph{Bezkontextová gramatika} je čtveřice $G = (V,T,P,S)$, kde $V$ je totální abeceda, $T \subseteq V$ je abeceda terminálů, $S \in (V-T)$ je startující symbol a~$P$ je konečná množina pravidel tvaru $q: A \to \alpha$, kde $A \in (V-T)$, $\alpha \in V^*$ a~$q$ je návěští tohoto pravidla. Nechť $N = V - T$ značí abecedu neterminálů. Pokud $q\colon A \to \alpha \in P$, $\gamma, \delta \in V^*$, $G$ provádí derivační krok z~$\gamma A \delta$ do $\gamma\alpha\delta$ podle pravidla $q\colon A \to \alpha$, symbolicky píšeme $\gamma A \delta \Rightarrow \gamma\alpha\delta\ [q\colon A \to \alpha]$ nebo zjednodušeně $\gamma A \delta \Rightarrow \gamma\alpha\delta$. Standardním způsobem definujeme $\Rightarrow^m$, kde $m \geq 0$. Dále definujeme tranzitivní uzávěr $\Rightarrow^+$ a~tranzitivně-reflexivní uzávěr $\Rightarrow^*$.
\end{define}
\par Algoritmus můžeme uvádět podobně jako definice textově, nebo využít pseudokódu vysázeného ve vhodném prostředí (například \texttt{algorithm2e}).
\begin{algor}
 \emph{Algoritmus pro ověření bezkontextovosti gramatiky. Mějme gramatiku G = (N, T, P, S).}
\begin{enumerate}
	\item \label{item1} \emph {Pro každé pravidlo $p \in P$ proveď test, zda $p$ na levé straně obsahuje právě jeden symbol z~$N$.}
	\item \emph{Pokud všechna pravidla splňují podmínku z~kroku \ref{item1}, tak je gramatika $G$ bezkontextová.} 

\end{enumerate}
\end{algor}
\begin{define}
\emph{Jazyk} definovaný gramatikou $G$ definujeme jako $L(G)=\{\omega \in T^*|S \Rightarrow^* \omega \}$.
\end{define}

\subsection{Podsekce obsahující větu} 
\begin{define}
Nechť $L$ je libovolný jazyk. $L$ je \emph{bezkontextový jazyk}, když a~jen když $L = L(G)$, kde $G$ je libovolná bezkontextová gramatika.
\end{define}
\begin{define}
Množinu $\mathcal{L}_{CF} = \{L|L$ je bezkontextový jazyk\} nazýváme \emph{třídou bezkontextových jazyků}.
\end{define}
\begin{sente} \label{sentenc1}
Nechť $L_{abc} = \{a^nb^nc^n|n \geq  0\} $. Platí, že $L_{abc} \notin \mathcal{L}_{CF}$.
\end{sente}

\begin{proof}
Důkaz se provede pomocí Pumping lemma pro bezkontextové jazyky, kdy ukážeme, že není možné, aby platilo, což bude implikovat pravdivost věty \ref{sentenc1}.
\end{proof}

\section{Rovnice a odkazy}

Složitější matematické formulace sázíme mimo plynulý text. Lze umístit několik výrazů na jeden řádek, ale pak je třeba tyto vhodně oddělit, například příkazem \verb|\quad| . 

$$ \displaystyle \sqrt[x^2]{y^3_0} \quad \mathbb{N} = \{0, 1, 2,\ldots\} \quad x^{y^y} \neq x^{yy} \quad z_{i_j} \not\equiv z_{ij}$$ 

V rovnici (\ref{rov1}) jsou využity tři typy závorek s~různou explicitně definovanou velikostí.

\begin{align} \label{rov1}
	\begin{split}
		\bigg\{\Big[\big(a+b\big)*c\Big]^d+1\bigg\} = x 
	\end{split}
\end{align}
	$$	\lim_{x \to \infty} \frac{\sin^2x + \cos^2x}{4} = y $$
	
V~této větě vidíme, jak vypadá implicitní vysázení limity $ \lim_{n \to \infty} f(n) $ v~normálním odstavci textu. Podobně je to i~s~dalšími symboly jako $ \sum_1^n $ či $\bigcup_{A \in \mathcal{B}} $ . V případě vzorce $\lim\limits_{x \to 0} \frac{\sin x}{x} = 1 $ jsme si vynutili méně úspornou sazbu příkazem \verb|\limits|.

\begin{eqnarray}
	\int\limits_a^b f(x)dx & = & -\int\limits_b^a f(x)dx \\
	\Big(\sqrt[5]{x^4}\Big)' = \Big(x^{\frac{4}{5}}\Big)' & = & \frac{4}{5}x^{-\frac{1}{5}} = \frac{4}{5\sqrt[5]{x}} \\
	\overline{\overline{A \vee B}}  & = & \overline{\overline{A} \wedge \overline{B}}
\end{eqnarray}

\section{Matice}

Pro sázení matic se velmi často používá prostředí \texttt{array} a~závorky (\verb|\left|, \verb|\right|). 

$$	\left(
	\begin{array}{cc}
	a + b & b-a \\
	\widehat{\xi + \omega} & \hat{\pi}\\
	\vec{a} & \overset\longleftrightarrow{AC} \\
	0 & \beta
	\end{array} \right)
 $$

$$
	A = \left\| 
	\begin{array}{cccc}
	a_{11} & a_{12} & \ldots & a_{1n} \\
	a_{21} & a_{22} & \ldots & a_{2n} \\
	\vdots & \vdots & \ddots & \vdots \\
	a_{m1} & a_{m2} & \ldots & a_{mn}
	\end{array} 	
	\right\|
$$

$$
	\left|
	\begin{array}{lr}
		t & u \\
		v & w
	\end{array}
	\right|	= tw - uv	
$$

Prostředí \texttt{array} lze úspěšně využít i~jinde.
$$  \binom nk =  \left\{
	\begin{array}{ll}
		\frac{n!}{k!(n-k)!} & \text{pro 0 $\leq k \leq $n}	\\
		0 & \text{pro k \textless 0  \mbox{nebo} k \textgreater n}	
	\end{array} \right. 	$$
	
\section{Závěrem}

V případě, že budete potřebovat vyjádřit matematickou konstrukci nebo symbol a~nebude se Vám dařit jej nalézt v samotném \LaTeX u, doporučuji prostudovat možnosti balíku maker \AmS-\LaTeX.
Analogická poučka platí obecně pro jakoukoli konstrukci v \TeX u.

\end{document}